%%%%%%%%%%%%%%%%%%%%%%%%%%%%%%%%%%%%%%%%%
% Memo
%
% The latex template has been downloaded from:
% http://www.LaTeXTemplates.com
%
% Original author:
% Rob Oakes (http://www.oak-tree.us) with modifications by:
% Vel (vel@latextemplates.com)
%
% License:
% CC BY-NC-SA 3.0 (http://creativecommons.org/licenses/by-nc-sa/3.0/)
%
%%%%%%%%%%%%%%%%%%%%%%%%%%%%%%%%%%%%%%%%%

\documentclass[a4paper,11pt]{texMemo} % Set the paper size (letterpaper, a4paper, etc) and font size (10pt, 11pt or 12pt)

\usepackage{amsmath}              %% permet de g�n�rer un index automatiquement
\usepackage{numprint}
\usepackage{parskip} % Adds spacing between paragraphs
\usepackage[babel=true]{csquotes} % pour les guillemets. csquotes va utiliser la langue d��finie dans babel
\setlength{\parindent}{15pt} % Indent paragraphs
\usepackage{numprint}
\usepackage[squaren,Gray]{SIunits} % pour les unit�s 

\graphicspath{{./figures/}}

%----------------------------------------------------------------------------------------
%	MEMO INFORMATION
%----------------------------------------------------------------------------------------

%\memoto{James Smith} % Recipient(s)

\memofrom{Laurent Ravera} % Sender(s)

\memosubject{Shape of current pulses in a TES in the case of critical damping} % Memo subject

\memodate{May 2, 2017} % Date, set to \today for automatically printing todays date

\logo{\includegraphics[width=0.25\textwidth]{logo-XIFU.png}} % logo at the top right of the memo

\bibliographystyle{unsrt-fr_LR}

%----------------------------------------------------------------------------------------

\begin{document}

\maketitle % Print the memo header information

%----------------------------------------------------------------------------------------
%	MEMO CONTENT
%----------------------------------------------------------------------------------------

The TES current profile after a temperature impulse is given by \cite{IrwinHilton2005}:

\begin{align}
I(t)\propto\frac{e^{-t/\tau_{+}}-e^{-t/\tau_{-}}}{1/\tau_{+}-1/\tau_{-}}
\end{align} 

$\tau_{+}$ and $\tau_{-}$ are respectively the \enquote{rise time constant} and the \enquote{fall time constant} of the TES current profile.


We have:

\begin{align}
I(t)\propto\frac{e^{-t/\tau_{+}}-e^{-t/\tau_{-}}}{1/\tau_{+}-1/\tau_{-}}= &\frac{e^{[-t(1/\tau_{+}-1/\tau_{-})-t/\tau_{-}]}-e^{[t(1/\tau_{+}-1/\tau_{-})-t/\tau_{+}]}}{1/\tau_{+}-1/\tau_{-}}\nonumber \\
= &\frac{e^{-t(1/\tau_{+}-1/\tau_{-})}e^{-t/\tau_{-}}-e^{t(1/\tau_{+}-1/\tau_{-})}e^{-t/\tau_{+}}}{1/\tau_{+}-1/\tau_{-}}
\label{eq_cur_prof}
\end{align} 

As $e^x\underset{x\to 0}{\longrightarrow} 1+x$, in the case of critical damping equation \ref{eq_cur_prof} becomes:

\begin{align}
\frac{e^{-t/\tau_{+}}-e^{-t/\tau_{-}}}{1/\tau_{+}-1/\tau_{-}}= &\frac{\left(1-t(1/\tau_{+}-1/\tau_{-})\right)e^{-t/\tau_{-}}-\left(1+t(1/\tau_{+}-1/\tau_{-})\right)e^{-t/\tau_{+}}}{1/\tau_{+}-1/\tau_{-}} \nonumber \\
= &-\frac{e^{-t/\tau_{+}}-e^{-t/\tau_{-}}}{1/\tau_{+}-1/\tau_{-}}-te^{-t/\tau_{-}}-te^{-t/\tau_{+}}\nonumber \\
= &-t/2\left(e^{-t/\tau_{-}}+e^{-t/\tau_{+}}\right)
\end{align} 

So the current profile in the case of critical damping ($\tau_{+}{\to} \tau_{-}$) is:
\begin{align}
I(t)\propto &-te^{-t/\tau_{\pm}}
\end{align} 

Figure \ref{fig_pulse} shows the TES current profile in the case of critical damping.

\begin{figure}[htp]
\begin{center}
\includegraphics[width=320pt]{pulse.png}
\caption{Shape of a current pulse in the case of critical damping with $\tau_{\pm}=$ \unit{286}{\micro\second}. Even if $\tau_{+}=\tau_{-}=\tau_{\pm}$, the pulse shape is not symmetric.}
\label{fcurrent_pulse_profile}
\label{fig_pulse}
\end{center}
\end{figure}

%----------------------------------------------------------------------------------------
\bibliography{biblio}

\end{document}