%%%%%%%%%%%%%%%%%%%%%%%%%%%%%%%%%%%%%%%%%
% Memo
%
% The latex template has been downloaded from:
% http://www.LaTeXTemplates.com
%
% Original author:
% Rob Oakes (http://www.oak-tree.us) with modifications by:
% Vel (vel@latextemplates.com)
%
% License:
% CC BY-NC-SA 3.0 (http://creativecommons.org/licenses/by-nc-sa/3.0/)
%
%%%%%%%%%%%%%%%%%%%%%%%%%%%%%%%%%%%%%%%%%

\documentclass[a4paper,11pt]{texMemo} % Set the paper size (letterpaper, a4paper, etc) and font size (10pt, 11pt or 12pt)

\usepackage{amsmath}              %% permet de g�n�rer un index automatiquement
\usepackage{numprint}
\usepackage{parskip} % Adds spacing between paragraphs
\usepackage[babel=true]{csquotes} % pour les guillemets. csquotes va utiliser la langue d��finie dans babel
\setlength{\parindent}{15pt} % Indent paragraphs
\usepackage{numprint}
\usepackage[squaren,Gray]{SIunits} % pour les unit�s 
\usepackage{cancel} % pour barrer du texte
\usepackage{multicol} % pour plusieurs equations sur la m�me ligne

\graphicspath{{./figures/}}

%----------------------------------------------------------------------------------------
%	MEMO INFORMATION
%----------------------------------------------------------------------------------------

%\memoto{James Smith} % Recipient(s)

\memofrom{Laurent Ravera} % Sender(s)

\memosubject{Impact of modulation and demodulation operations in AC-readout} % Memo subject

\memodate{May 29, 2017} % Date, set to \today for automatically printing todays date

\logo{\includegraphics[width=0.25\textwidth]{logo-XIFU.png}} % logo at the top right of the memo

\bibliographystyle{unsrt-fr_LR}

%----------------------------------------------------------------------------------------

\begin{document}

\maketitle % Print the memo header information

%----------------------------------------------------------------------------------------
%	MEMO CONTENT
%----------------------------------------------------------------------------------------

\section{Demodulation of the signal}

At the BBFB firmware input, the carrier signal $\cos\left(\omega t + \phi\right)$ is amplitude modulated by the "science signal" $S(t)$. To do the demodulation we multiply this input signal by the complex signal $e^{-i\omega t} $ (same frequency as the AC-carrier) as follows:

\begin{align}
\cos\left(\omega t + \phi\right)\times S(t)\times e^{-i\omega t} 
=&\left\{ 
	\begin{array}{ll}
		\cos\left(\omega t + \phi\right)\times S(t)\times \cos\left(\omega t\right)\\
		-\cos\left(\omega t + \phi\right)\times S(t)\times \sin\left(\omega t\right)\\
	\end{array}
	\right.\\
=&\left\{ 
	\begin{array}{ll}
		S(t)\times \frac{1}{2}\times \left[\cancel{\cos\left(2\omega t + \phi\right)}+cos\left(\phi\right)\right]\\
		-S(t)\times \frac{1}{2}\times \left[\cancel{\sin\left(2\omega t + \phi\right)}+sin\left(-\phi\right)\right]\\
	\end{array}
	\right. \label{equ_neglect} \\ 
=&\left\{ 
	\begin{array}{ll}
		\frac{S(t)}{2}\times cos\left(\phi\right)\\
		\frac{S(t)}{2}\times sin\left(\phi\right)\\
	\end{array}
	\right.
\end{align} 


The BBFB firmware computes the module of the demodulated signal, which is:

\begin{align}
\left|S^d(t)\right|=\sqrt{I^2+Q^2}=\frac{S(t)}{2}
\end{align} 

The power of the science signal has been divided by four in this process: 
\begin{itemize}
\item
A factor of two comes from AC-type modulation in the TES: modulating a sine wave produces a loss by a factor of two in power compared to modulating a DC-signal.
\item
An other factor of two comes from the demodulation and the filtering of the high frequencies (in equation \ref{equ_neglect}). 
\end{itemize}


\section{Demodulation of the noise}

We distinguish two kind of noise: the noise from the TES $N_{TES}(t)$ and the noise from the amplification chain $N(t)$.


The noise from the TES is band-limited so that the demodulation has exactly the same impact on $N_{TES}(t)$ than on $S(t)$:
\begin{align}
\left|N^d_{TES}(t)\right|=\frac{N_{TES}(t)}{2}
\end{align} 


The noise from the amplification chain is white so that the shift to the high frequencies does not allow to neglect the terms in $2\omega t$ in the equation \ref{equ_neglect}. The loss of power by a factor of two is not applicable in this case and thus we have:
\begin{align}
\left|N^d(t)\right|=\frac{N(t)}{\sqrt{2}}
\end{align} 


\section{Comparison with DC-type readout}


In the case of a DC-type biasing, the signal and the noises obtained after the readout directly: 

\begin{multicols}{3}\noindent
\begin{equation}S^d(t) = S(t) \nonumber \end{equation} 
\begin{equation}N_{TES}^d(t)=N_{TES}(t) \nonumber \end{equation}
\begin{equation}N^d(t)=N(t)\end{equation}
\end{multicols}


%----------------------------------------------------------------------------------------
\bibliography{biblio}

\end{document}